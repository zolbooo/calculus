\documentclass{article}
\usepackage[utf8]{inputenc}
\usepackage[english,russian]{babel}

\usepackage{amsmath,amsthm,amssymb}
\usepackage{mathtext,mathtools}

\author{
	Гантумур, Золбообаяр,\\
	z-gantomor@stud.kpfu.ru
}
\title{Контрольная работа №1, вариант VI}
\date{12 марта 2023}

\begin{document}
	\maketitle
	% First problem
	$$1. \int\frac{5x-13}{(x^2-5x+6)^2}dx = \int\frac{5x-13}{(x-3)^2(x-2)^2}dx =$$
	$$= \int(\frac{A}{x-3} + \frac{B}{(x-3)^2} + \frac{C}{x-2} + \frac{D}{(x-2)^2})dx =$$
	$$= \int\frac{A}{x-3}dx + \int\frac{B}{(x-3)^2}dx + \int\frac{C}{x-2}dx + \int\frac{D}{(x-2)^2}dx =$$
	$$= A\int\frac{d(x - 3)}{x-3} + B\int\frac{d(x - 3)}{(x-3)^2} + C\int\frac{d(x - 2)}{x - 2} + D\int\frac{d(x - 2)}{(x-2)^2} =$$
	$$= A\ln|x - 3| - \frac{B}{x - 3} + Cln|x - 2| - \frac{D}{x - 2} + c$$

	$$\frac{(x-3)(x-2)^2A + (x-2)^2B + (x-2)(x-3)^2C + (x-3)^2D}{(x-3)^2(x-2)^2} =$$
	\begin{flalign*}
		& x^3: A + C = 0\\
		& x^2: -5A + B - 5C + D = 0\\
		& x: 10A - 2B + 15C - 3D = 5\\
		& 1: -12A + 4B - 18C + 9D = 13
	\end{flalign*}

	\begin{equation*}
		\begin{cases}
			A + C = 0\\
			-5A + B - 5C + D = 0 \Rightarrow B + D = 0\\
			10A - 2B + 15C - 3D = 5 \Rightarrow -2B + 5C - 3D = 5\\
			-12A + 4B - 18C + 9D = 13 \Rightarrow 4B - 6C + 9D = 13
		\end{cases}
	\end{equation*}
	\begin{equation*}
		\begin{cases}
			A + C = 0\\
			B + D = 0\\
			-2B + 5C - 3D = 5 \Rightarrow 5C - D = 5\\
			4B - 6C + 9D = 13 \Rightarrow -6C + 5D = 13
		\end{cases}
	\end{equation*}
	\begin{equation*}
		\begin{cases}
			A + C = 0\\
			B + D = 0\\
			5C - D = 5 \Rightarrow 30C - 6D = 30\\
			-6C + 5D = 13 \Rightarrow -30C + 25D = 65 \Rightarrow 19D = 95
		\end{cases}
	\end{equation*}
	$$D = 5 \Rightarrow B = -5$$
	$$C = 2 \Rightarrow A = -2$$

	\emph{Ответ.} $$\int\frac{5x-13}{(x^2-5x+6)^2}dx = -2\ln|x - 3| + \frac{5}{x - 3} + 2ln|x - 2| - \frac{5}{x - 2} + C$$
	% End of first problem
	% Problem 2
	$$2. \int \frac{dx}{(x-1)^2(x^2+4)^2}$$
	Решение методом Остроградского.
	$$Q(x) = (x-1)^2(x^2+4)^2 \Rightarrow (Q(x))' = 2(x-1)(x^2+4)^2 + 4x(x^2+4)(x-1)$$
	$$(Q(x))' = 2(x-1)(x^2+4)(x^2+2x+4)$$
	$$Q_1(x) = Q_2(x) = (x-1)(x^2+4)$$

	$$\int\frac{dx}{(x-1)^2(x^2+4)^2} = \frac{Ax^2 + Bx + C}{(x-1)(x^2+4)} + \int\frac{Dx^2 + Ex + F}{(x-1)(x^2+4)}dx$$

	$$\frac{(2Ax + B)(x^3 - x^2 + 4x - 4)}{(x-1)^2(x^2+4)^2} +$$
	$$+ \frac{(3x^2 - 2x + 4)(Ax^2+Bx+C)}{(x-1)^2(x^2+4)^2} +$$
	$$+ \frac{(x - 1)(x^2+4)(Dx^2 + Ex + F)}{(x-1)^2(x^2+4)^2} = \frac{1}{(x-1)^2(x^2+4)^2}$$

	$$x^5: D = 0$$
	$$x^4: 2A + 3A + E - D = 5A + E = 0$$
	$$x^3: -2A + B + 3B - 2A + F - E + 4D = -4A + 4B - E + F = 0$$
	$$x^2: 8A - B + 4A - 2B + 3C - F + 4E - 4D = 12A -3B + 3C + 4E - F = 0$$
	$$x: -8A + 4B + 4B - 2C - 4F - 4E = -4A + 4B - C - 2E - 2F = 0$$
	$$1: 4B + 4C - 4F = 1$$
	\begin{equation*}
		\begin{cases}
			5A + E = 0 \Rightarrow E = -5A\\
			-4A + 4B - E + F = 0 \Rightarrow A + 4B + F = 0\\
			12A - 3B + 3C + 4E - F = 0 \Rightarrow -8A - 3B + 3C - F = 0\\
			-4A + 4B - C - 2E + 2F = 0 \Rightarrow 6A + 4B - C + 2F = 0\\
			4B + 4C - 4F = 1
		\end{cases}
	\end{equation*}
	\begin{equation*}
		\begin{cases}
			A + 4B + F = 0 \\
			-8A - 3B + 3C - F = 0 \Rightarrow 29B + 3C + 7F = 0\\
			6A + 4B - C + 2F = 0 \Rightarrow -20B - C - 4F = 0\\
			4B + 4C - 4F = 1
		\end{cases}
	\end{equation*}
	\begin{equation*}
		\begin{cases}
			-20B - C - 4F = 0\\
			29B + 3C + 7F = 0 \Rightarrow -31B + 5F = 0\\
			4B + 4C - 4F = 1 \Rightarrow -76B - 20F = 1
		\end{cases}
	\end{equation*}
	\begin{equation*}
		\begin{cases}
			-31B + 5F = -124B + 20F = 0\\
			-76B - 20F = 1 \Rightarrow -200B = 1
		\end{cases}
	\end{equation*}
	\begin{equation*}
		\begin{cases}
			B = -\frac{1}{200}\\
			-31B + 5F = 0 \Rightarrow 5F = \frac{31}{200} \Rightarrow F = \frac{31}{1000}\\
			-20B - C - 4F = 0 \Rightarrow C = -20B - 4F \Rightarrow C = \frac{1}{10} - \frac{31}{250} = -\frac{3}{125}\\
			A + 4B + F = 0 \Rightarrow A = -4B - F = \frac{1}{50} - \frac{31}{1000} = -\frac{11}{1000}\\
			E = -5A \Rightarrow E = \frac{11}{200}
		\end{cases}
	\end{equation*}
	\begin{equation*}
		\begin{cases}
			A = -\frac{11}{1000}\\
			B = -\frac{1}{200}\\
			C = -\frac{3}{125}\\
			D = 0\\
			E = \frac{11}{200}\\
			F = -\frac{11}{1000}
		\end{cases}
	\end{equation*}

	$$\int\frac{Dx^2 + Ex + F}{(x-1)(x^2+4)}dx = \int\frac{\frac{11}{200}x -\frac{11}{1000}}{(x-1)(x^2+4)}dx = \frac{11}{200}\int\frac{x-\frac{1}{5}}{(x-1)(x^2+4)}dx =$$
	$$= \frac{11}{200}\int\frac{(x - 1) + \frac{4}{5}}{(x-1)(x^2+4)}dx = \frac{11}{200}(\int\frac{dx}{x^2+2^2} + \frac{4}{5}\int\frac{dx}{(x-1)(x^2+4)})$$

	$$\int\frac{dx}{x^2+2^2} = \frac{1}{2}\arctg{\frac{x}{2}} + C$$

	$$\int\frac{dx}{(x - 1)(x^2 + 4)} = \int\frac{A}{x - 1}dx + \int\frac{Bx + C}{x^2 + 4}dx$$
	$$\frac{A}{x - 1} + \frac{Bx + D}{x^2 + 4} = \frac{(x^2+4)A + (x-1)(Bx+C)}{(x-1)(x^2+4)}$$
	$$x^2: A + B = 0$$
	$$x: -B + C = 0$$
	$$1: 4A - C = 1$$
	\begin{equation*}
		\begin{cases}
			A + B = 0 \Rightarrow A = -B\\
			-B + C = 0 \Rightarrow B = C\\
			4A - C = 1 \Rightarrow -4B - C = 1 \Rightarrow -5B = 1
		\end{cases}
	\end{equation*}
	\begin{equation*}
		\begin{cases}
			B = C = -\frac{1}{5}\\
			A = \frac{1}{5}
		\end{cases}
	\end{equation*}

	$$\int(\frac{A}{x - 1} + \frac{Bx + C}{x^2 + 4})dx = \frac{1}{5}\int\frac{dx}{x - 1} - \frac{1}{5}\int\frac{x + 1}{x^2 + 4}dx$$
	$$\int\frac{x + 1}{x^2 + 4}dx = \int\frac{x}{x^2+4} + \int\frac{1}{x^2 + 4} = \frac{1}{2}\ln(x^2 + 4) + \frac{1}{2}\arctg{\frac{x}{2}} + C$$
	$$\int\frac{x}{x^2 + 4}dx = \frac{1}{2}\int\frac{dt}{t} = \frac{1}{2}\ln|t| + C = \frac{1}{2}\ln(x^2 + 4) + C$$
	$$\frac{1}{5}\int\frac{dx}{x - 1} - \frac{1}{5}\int\frac{x + 1}{x^2 + 4}dx = \frac{1}{5}\ln|x - 1| + \frac{1}{2}\ln(x^2 + 4) + \frac{1}{2}\arctg{\frac{x}{2}} + C$$

	\emph{Ответ.} $$\int\frac{dx}{(x-1)^2(x^2+4)^2} = \frac{\frac{11}{40}x^2 - \frac{1}{8}x - \frac{3}{5}}{25(x-1)(x^2+4)} + \frac{11}{200}(\frac{1}{2}\arctg{\frac{x}{2}} + \frac{4}{5}(\frac{1}{5}\ln|x - 1| + \frac{1}{2}\ln(x^2 + 4) + \frac{1}{2}\arctg{\frac{x}{2}})) + C$$
	% End of problem 2
	% Problem 3
	$$3. \int\sqrt{(x^2+1)^3}dx = \int(x^2 + 1)\sqrt{x^2 + 1}dx$$
	$$u = \sqrt{x^2 + 1} \Rightarrow du = \frac{xdx}{\sqrt{x^2 + 1}}$$
	$$dv = (x^2 + 1)dx \Rightarrow v = \frac{x^3}{3} + x$$
	$$\int(x^2 + 1)\sqrt{x^2 + 1}dx = \frac{\sqrt{x^2 + 1}(x^3 + 3x)}{3} - \int\frac{x^4 + 3x^2}{3\sqrt{x^2 + 1}}dx$$

	$$t = x^2 + 1 \Rightarrow dt = 2xdx$$
	$$\int\frac{x^4 + 3x^2}{3\sqrt{x^2 + 1}}dx = \int\frac{\sqrt{t - 1}(t + 2)}{6t}dt = \frac{1}{6}\int\sqrt{t - 1}dt + \int\frac{\sqrt{t - 1}}{3t}dt$$
	$$\int\sqrt{t - 1}dt = \frac{3\sqrt{t - 1}}{2} + C = \frac{3|x|}{2} + C$$

	$$\int\frac{\sqrt{t - 1}}{3t}dt$$
	$$u = \frac{1}{3t} \Rightarrow du = \frac{ln|t|}{3}dt$$
	$$dv = \sqrt{t - 1}dt \Rightarrow v = \frac{3\sqrt{t - 1}}{2}$$

	$$\int\frac{\sqrt{t - 1}}{3t}dt = \frac{\sqrt{t - 1}}{2t} - \frac{1}{2}\int\sqrt{t - 1}ln|t|dt$$
	$$u = ln|t| \Rightarrow du = \frac{1}{t}dt$$
	$$dv = \sqrt{t - 1}dt \Rightarrow v = \frac{3\sqrt{t - 1}}{2}$$
	$$\int\sqrt{t - 1}ln|t|dt = \frac{3\sqrt{t - 1}ln|t|}{2} - \frac{9}{2}\int\frac{\sqrt{t - 1}}{3t}dt$$
	$$\int\frac{\sqrt{t - 1}}{3t}dt = \frac{\sqrt{t - 1}}{2t} - \frac{3}{4}\sqrt{t - 1}ln|t| + \frac{9}{4}\int\frac{\sqrt{t - 1}}{3t}dt$$
	$$-\frac{5}{4}\int\frac{\sqrt{t - 1}}{3t}dt = \frac{\sqrt{t - 1}}{2t} - \frac{3}{4}\sqrt{t - 1}ln|t|$$
	$$\int\frac{\sqrt{t - 1}}{3t}dt = -\frac{2\sqrt{t - 1}}{5t} + 15\sqrt{t - 1}ln|t|$$
	% t = x^2 + 1
	$$\int\frac{\sqrt{t - 1}(t + 2)}{6t}dt = \frac{|x|}{4} + -\frac{2|x|}{5(x^2 + 1)} + 15|x|ln(x^2 + 1) + C$$
	\emph{Ответ.} $$\int\sqrt{(x^2+1)^3}dx = \frac{\sqrt{x^2 + 1}(x^3 + 3x)}{3} - \frac{|x|}{4} + -\frac{2|x|}{5(x^2 + 1)} + 15|x|ln(x^2 + 1) + C$$
	% End of problem 3
	% Problem 4
	$$4. \int\frac{dx}{(1+x)\sqrt{x-x^2}}$$
	$$u = \frac{1}{1 + x} \Rightarrow du = -\frac{dx}{(1 + x)^2}$$
	$$\int\frac{dx}{\sqrt{x-x^2}} = \int\frac{dx}{\sqrt{-(x^2 - x + (\frac{1}{2})^2) + (\frac{1}{2})^2}} = \int\frac{dx}{\sqrt{-(x - \frac{1}{2})^2 + (\frac{1}{\sqrt{2}})^2}}$$
	$$\int\frac{dx}{\sqrt{(\frac{1}{\sqrt{2}})^2 - (x - \frac{1}{2})^2}} = \int\frac{dt}{\sqrt{(\frac{1}{\sqrt{2}})^2 - t^2}} = \arcsin(2x - 1) + C$$
	$$dv = \frac{dx}{\sqrt{x-x^2}} \Rightarrow v = \arcsin(2x - 1)$$

	$$\int\frac{dx}{(1+x)\sqrt{x-x^2}} = \frac{\arcsin(2x - 1)}{1 + x} - \int\frac{\arcsin(2x - 1)}{(1 + x)^2}dx$$
	$$u = arcsin(2x - 1) \Rightarrow du = \frac{2}{\sqrt{1 - (2x - 1)^2}}dx$$
	$$dv = \frac{1}{(1 + x)^2}dx \Rightarrow v = -\frac{1}{1 + x}$$
	$$\int\frac{\arcsin(2x - 1)}{(1 + x)^2}dx = -\frac{arcsin(2x - 1)}{1 + x} + \int\frac{2}{(1 + x)(1 - (2x - 1)^2)}dx$$

	$$\int\frac{2}{(1 + x)(1 - (2x - 1)^2)}dx = \frac{1}{2}\int\frac{1}{x(1 + x)(1 - x)}dx$$
	$$\int\frac{1}{x(1 + x)(1 - x)}dx = \int\frac{(1 + x)(1 - x)A + x(1 - x)B + x(1 + x)C}{x(1 + x)(1 - x)}dx$$
	$$x^2: -A - B + C = 0$$
	$$x: -A + A + B + C = B + C = 0$$
	$$1: A = 1$$
	\begin{equation*}
		\begin{cases}
			A = 1\\
			-B + C = 1\\
			B + C = 0 \Rightarrow 2C = 1 \Rightarrow C = \frac{1}{2}\\
			B = -\frac{1}{2}
		\end{cases}
	\end{equation*}

	$$\int\frac{1}{x(1 + x)(1 - x)}dx = \int\frac{1}{x}dx - \frac{1}{2}\int\frac{1}{1 + x}dx + \frac{1}{2}\int\frac{1}{1 - x}dx =$$
	$$= \ln|x| - \frac{1}{2}\ln|x + 1| - \frac{1}{2}\ln|1 - x| + C$$

	\emph{Ответ.} $$\int\frac{dx}{(1+x)\sqrt{x-x^2}} = \frac{\arcsin(2x - 1)}{1 + x} - \ln|x| + \frac{1}{2}\ln|x + 1| + \frac{1}{2}\ln|1 - x| + C$$
	% End of problem 4
\end{document}
